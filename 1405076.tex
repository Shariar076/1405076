\documentclass{report}
\usepackage{color}
\renewcommand{\thechapter}{\Roman {chapter}}

\begin{document}
\title{A Report on The Element of Style}
\author{MD. Shariar Kabir\\Student ID:1405076}
\date{April 20, 2017}
\maketitle

\tableofcontents{}

\chapter{\color{blue} Introduction}
This Book gives us an overview of \textbf{Elements of style} and almost all of the rules in the main book is covered here with some possible exceptions.
Chapter \ref{chap2} of this book covers elementary rules of usage.Chapter \ref{chap3} discusses elementary rules of composition.Chapter \ref{chap4} deals with few matter of form. The common mistakes in use are discussed in chapter \ref{chap5}. Finally the last chapter lists some commonly misspelled words.  

	

\chapter{\color{blue}The Elements of Style}
\label{chap2}
\begin{enumerate}
\item \textbf{Formation of possesive singular of nouns by adding 's}\\
We use this rule whatever the final consonant is. Such as,
\begin{itemize}
\item Danny's book.
\item Alex's doll.
\end{itemize}

Except the possessives of ancient proper name that end with -es or -is, in these cases we need to add an apostrophe followed by an s to create the possessive form
\begin{itemize}
\item The Pepins' house is right on the corner.
\item The witches' broom was hidden in th closet.
\item The babies' beds are all in a row.
\end{itemize}

However pronominal possessives like hers, its , theirs has no apostrophe, we use apostrophe for indefinite pronouns to show possession.
\begin{itemize}
	\item Someone else's book.
\end{itemize}

\item \textbf{Use of comma in a series of three or more terms}\\
In a series of three or more terms with a single conjunction, we use a comma after each term except the last. It's called the serial comma.
\begin{itemize}
	\item She had an apple, an orange and two bananas.
\end{itemize}
Usually in case of business firms the last comma is omitted  
\begin{itemize}
	\item Brown,Shipley and Company
\end{itemize}

\item textbf{Enclosing parenthetic expressions between commas}\\
For example we write,
\begin{itemize}
	\item The best way to enjoy life,if you have plenty time and money, is  to go on tours.
\end{itemize}
However if the interruption to the flow of the sentence is slight, we can safely omit \textbf{both} commas. But whether the interruption is slight or considerable we must never omit one comma leave the other one. This may cause error or misunderstanding. Like
\begin{itemize}
	\item My friend if only you weren't so desperate, you might have won.
\end{itemize}

\item \textbf{Placing a comma before a conjunction introducing an independent clause}\\
\begin{itemize}
	\item The situation of the patient is serious, but there is still hope.
\end{itemize}
A comma is useful when the subject is the same for both clauses and is expressed only once and if the connective is but. However when the connective is and, the comma should be omitted if the relation between the two statements is close.
\begin{itemize}
	\item He is an experienced and very talented worker.
\end{itemize}

\item \textbf{Do not join independent clauses with a comma}\\
It is more convenient to use a semicolon if two or more clauses are not joined by a conjunction and are grammatically complete. 
\begin{itemize}
	\item It is getting dark; we should look for a place to spend the night.
\end{itemize}
But if there's a conjunction, the proper mark is a comma. Also it is an exception if the clauses are very short and alike or when the tone is easy and conversational, it is more preferable to use a comma.
\begin{itemize}
	\item He came to my house yesterday, took his stuffs and left forever.
	\item Man proposes, God disposes.
\end{itemize}

\item \textbf{Do not break the sentence in two}\\
We must not use a period in place of commas and divide the sentence.
\begin{itemize}
	\item He is an excellent teacher. A man with a strong voice and clear concept of the subject.
\end{itemize}
In this example the second sentence doesn't make much sense if we use a period and divide the sentence in two.

\item \textbf{A participial phrase at the beginning of a sentence must refer to the grammatical subject}\\
Participle phrases preceded by a conjunction or preposition, nouns in apposition,adjective ,adjective phrase follow this rule.
\begin{itemize}
	\item Standing at the riverbank, I saw the boats sailing through the waves.
\end{itemize}

\item \textbf{Divide words at line-ends, in accordance with their formation and pronunciation }\\
If there's space at the end of a line for one or more syllables but not for the whole word, we can divide the word in to two, unless it involves cutting it only by a single letter. Although there are no hard and fast rules for the cutting the following principles are frequently applicable
\textbf{Divide the word according to its formation: }\\
	\begin{enumerate}
		\item know-ledge (not knowl-edge)
		\item atmo-sphere  (not atmos-phere)
	\end{enumerate}
\textbf{Divide on the vowel: }\\
	\begin{enumerate}
		\item ordi-nary 
		\item espe-cially
	\end{enumerate}
\textbf{Divide between double letters, unless they come at the
	end of the simple form of the word: }\\
	\begin{enumerate}
		\item omit-ting 
		\item swim-ming
	\end{enumerate}
\end{enumerate}



\chapter{\color{blue}Elementary Principles of Composition}
\label{chap3}
\begin{enumerate}
	
	\item \textbf{Make the paragraph the unit of composition: one paragraph to each topic}\\
	
	If we intend to discuss a what we are writing very briefly there may be no need of subdividing it into topics. However, a subject that requires subdivision into topics, each of the topics should be made the subject of a paragraph. The beginning of each paragraph is a signal to the reader that a new step in the development of the subject has been reached.\\
	The extent of the subdivision varies with the length of the composition. For example,\\
	A short notice of a book or a poem may consist of a single paragraph, but a slightly longer may consist of two paragraphs:\\
	\begin{enumerate}
		\item Account of the work.
		\item Critical discussion.
	\end{enumerate}
	A report on a poem might consist of seven paragraphs.\\
	\begin{enumerate}
		\item  Facts of composition and publication.
		\item Kind of poem; metrical form.
		\item Subject.
		\item Treatment of subject.
		\item For what chiefly remarkable.
		\item Wherein characteristic of the writer.
		\item Relationship to other works.
	\end{enumerate}
	
	A novel might be discussed under the heads
	\begin{enumerate}
		\item Setting.
		\item Plot.
		\item Characters.
		\item Purpose.
	\end{enumerate}
	
	A historical event might be discussed under the heads
	\begin{enumerate}
		\item What led up to the event.
		\item Account of the event.
		\item Results of the event.
	\end{enumerate}
	
	\item \textbf{Begin each paragraph with a topic sentence; end it in conformity with the beginning}\\
	The most useful kind of paragraph typically consists of three parts 
	\begin{enumerate}
		\item the topic sentence comes at or near the beginning
		\item the succeeding sentences explain or establish or develop the statement made in the topic sentence; and
		\item the final sentence either emphasizes the thought of the topic sentence or states some important consequence.
	\end{enumerate}
This practice helps a reader to easily discover the purpose of paragraph
when he starts to read it and end up with the same idea given at the
beginning of the paragraph.
Example:
	\begin{table}[h]
		\centering
		\begin{tabular}{|p{10cm}|c|}
			\hline
		 Global Warming means gradual increase in world’s temperature caused by greenhouse gases. & Topic sentence\\
		 \hline
 		The impact of global warming can be seen in sea level, crops, rainfall, and human health.Massive deforestation, burning of fossil fuels, industrial emissions, etc. have resulted to an increase in green-house gases around earth’s atmosphere. Massive deforestation, burning of fossil fuels, industrial emissions, etc. have resulted to an increase in green-house gases around earth’s atmosphere. & Cause and effect\\
 		\hline 
 		To stop global warming we should stop deforestation and adopt afforestation and other forest conservation methods.
		 We must also reduce the use of fossil fuels in power and electricity generation.  & Measured to be taken \\
		 \hline
		In conclusion it can be said that global warming is a threat to life of all kind. So it is high time we take necessary measures to stop this. & Conclusion\\
		\hline
		\end{tabular}
	\end{table}
	
	\item \textbf{Use the active voice}\\
	Using active voice is more direct and vigorous than the passive:
	\begin{itemize}
		\item Honey tastes sweet.
	\end{itemize}
	Expressing it in this way is much better than
	\begin{itemize}
		\item Honey is sweet when it is tasted.
	\end{itemize}
	
	This doesn't mean that the writer should entirely discard the passive voice. Sometimes passive voice is frequently convenient and necessary. 
	
	\begin{itemize}
		\item Every year thousands of people are killed on our roads.
	\end{itemize}
	
	\item \textbf{Put statements in positive form}\\
	We should avoid tame, colorless, hesitating, non-committal language and make definite assertions. Use the word \textit{not} as a means of denial or in antithesis, never as a means of evasion.
	\begin{table}[h]
		\centering
		\begin{tabular}{|c|c|}
			\hline
			He doesn't very often speaks the truth. & He usually lies. \\	
			\hline
			He did not think working hark is useful & He though hardworking is useless. \\
			\hline	
		\end{tabular}
	\end{table}
	
	\item \textbf{Omit needless words}\\
	
	A sentence shouldn't have unnecessary words just as a drawing shouldn't have unnecessary lines and a machine shouldn't have unnecessary parts.
	
	\begin{itemize}
		\item \textbf{Unnecessary: }in a nice way
		\item \textbf{Correct: }nicely
		\item \textbf{Unnecessary: }this is a subject that
		\item \textbf{Correct: }this subject
	\end{itemize}

	This doesn't means that the writer should make all his sentences short, or that he avoid all detail
	\item \textbf{Avoid a succession of loose sentences}\\
	Sentences of particular type should , those consisting  of
	two co-ordinate clauses, the second introduced by a conjunction or relative. Although
	single sentences of this type may be unexceptionable, a series soon
	becomes monotonous and tedious.
	
	Also constructing a whole paragraph of sentences of this kind, using as connectives and, but, and less frequently, who, which, when, where, and
	while is unskillful.
	
	\begin{itemize}
		\item \textbf{Unskillful: }Mr. Malik became the headmaster, and he was elected enormously.So he started his job as a headmaster ,and he was very responsible.The college students support his work,and he was happy.
	\end{itemize}
	
\item \textbf{Express co-ordinate ideas in similar form}\\
	
	This principle requires that expressions of similar content and function should
	be outwardly similar. The likeness of form enables the reader to recognize more
	readily the likeness of content and function.
	It is true that in repeating a statement in order to emphasize it he may have
	need to vary its form. For illustration,
	
	\begin{itemize}
		\item \textbf{incorrect: }Formerly, science was taught by the textbook method, while now the laboratory method is employed.
		\item \textbf{Correct:} Formerly, science was taught by the textbook method; now it is taught by the laboratory method.
	\end{itemize}
	By this principle, an article or a preposition applying to all the members of a
	series must either be used only before the first term or else be repeated before
	each term
	
	\item \textbf{Keep related words together}\\
	
	The writer must bring together the words, and groups of words, that are related
	in thought, and keep apart those which are not so related.
	The subject of a sentence and the principal verb should not, as a rule, be
	separated by a phrase or clause that can be transferred to the beginning.
	\begin{itemize}
		\item \textbf{Incorrect: }Bangladesh,in the Liberation War, lost thousands of fighters.
		\item \textbf{Correct:} In Liberation war,Bangladesh lost thousands of fighters.
	\end{itemize}
	\item \textbf{Keep summaries in one tense}\\
	If the summary is in the present tense, antecedent action should be expressed by the perfect; if in the past, by the past perfect.
	\begin{itemize}
		\item The poem, \textbf{‘Kobor’} deals with the immortality of creative people who has worked
		for the development of society.Man is mortal but he can live among the
		minds of thousands by his noble deeds. What he has done to the people,is
		remembered forever.
	\end{itemize}

this is written in present tense.
\end{enumerate}


\chapter{\color{blue}A few matters of form}
\label{chap4}
\begin{enumerate}
	\item \textbf{Headings:}
	After the title or heading of a manuscript, leave a blank line. On succeeding
	pages, if using ruled paper,begin on the first line
	\item \textbf{Numerals:}
	dates or othe serial numbers need not to be spelled. For example
	\begin{enumerate}
		\item \textbf{June 2,2011}
		\item \textbf{Rule 4}
	\end{enumerate}
	\item \textbf{Parentheses:}
	The reader should read as if the expression were absent,if there is any sentence
	or expression in parenthesis. But the final stop will not be omitted if there is
	any question mark or exclamation mark.
	\begin{itemize}
		\item He again proved that(for the second time) he can not be defeated.
		\item The little boy cried out(again and again) for food.
	\end{itemize}
	\item \textbf{Quotations:}
	Use colon before any quotation that has resulted from any documentary proof.
	\begin{itemize}
		\item The demand of Bangladesh people is:“No Rampal in Sundarbans”
	\end{itemize}
	
	\item \textbf{References:}
	References are very important things for writings. You should practice to give reference in parenthesis or in footnote,but never give references in the main body of the sentence. Writer should omit the words \textbf{act, scene, line, book, volume, page, except} when referring by only one of them.
	
	\begin{itemize}
		\item \textbf{Rikter Bedon 33-34}
		\item \textbf{Biddhostho Nilima 40-45}
	\end{itemize}
	
	\item \textbf{Titles:}
	Titles are very important. For the titles of literary works,scholarly usage prefers italic with capitalized initials.Don’t use ‘A’ or ‘The’ initially if you use any possessive before them.
		\begin{itemize}
		\item \textbf{Game of Thrones}
		\item \textbf{Romeo and Juliet}
	\end{itemize}
	
\end{enumerate}

\chapter{\color{blue}Words and expressions commonly misused}
\label{chap5}
\begin{enumerate}
	\item \textbf{Character}\\
	The word \textit{‘character’} is often used redundant, where the writer becomes a
	victim to mere habit of wordiness
	\begin{itemize}
		\item \textbf{Incorrect: } Behaviour of a cruel character.
		\item \textbf{Correct:} Cruel behaviour.
	\end{itemize}
	\item \textbf{A man who}\\
	Don’t use redundant expressions like \textit{‘a man who’},\textbf{‘a person who’},\textit{‘a boy who’} etc. where it is unnecessary.
	\begin{itemize}
		\item \textbf{Incorrect: } Md.Kaykobad sir is a man who is famous for ‘Algorithms’.
		\item \textbf{Correct:} Md.kaykobad sir is famous for ‘Algorithms’.
	\end{itemize}
	\item \textbf{Less}
	When less is written to express quality or number,it is wrong.So we have to use
	fewer for number.
	\begin{itemize}
		\item \textbf{Incorrect: } He has less money to buy book.
		\item \textbf{Correct:} He has fewer money to buy book.
	\end{itemize}
	Exception, when less is used as less quantity or amount,its correct.
	\begin{itemize}
		\item He has less than ten taka.
	\end{itemize}

	\item \textbf{Bid}\\
	Don’t use textit{'bid'} to,only use bid whose past tense is bade.
	\begin{itemize}
		\item I bid him goodnight.
		\item The Headmaster bade us farewell.
	\end{itemize}

	\item \textbf{Possess}\\
	Don’t use it since there are subtitles for \textit{'have'} or \textit{'own'}
	\begin{itemize}
		\item \textbf{Incorrect: } He owns a big house.
		\item \textbf{Correct:} He possess a big house.
	\end{itemize}

	\item \textbf{Compare}\\
	To show similarity between different objects \textit{‘compare to’} is used. again to show difference between similar objects \textit{‘compare with’} is used.
	\begin{itemize}
		\item Kazi Nazrul Islam is compared with Shakespeare.
	\end{itemize}

	\item \textbf{Most}\\
	\textit{Most} indicates larger amount.But often is it misused in place of \textit{almost},which
	means approximately
	\begin{itemize}
		\item \textbf{Incorrect: } Most everyday
		\item \textbf{Correct:} Almost everyday
	\end{itemize}
	
	\item \textbf{System}\\
	Do not use ‘system’ in redundant case.
	\begin{itemize}
		\item \textbf{Incorrect: } Course management system
		\item \textbf{Correct:} Course management
	\end{itemize}

	\item \textbf{State}\\
	Don’t use replacing say or remark.It is restricted to the sense of express fully or clearly.
	\begin{itemize}
		\item State the Newton’s law and prove it.
	\end{itemize}
	\item \textbf{All right}\\
	Don’t use it unless you want to mean ‘agreed’ or ‘go ahead’.Write it as separate two words.
	\begin{itemize}
		\item All right, I will go there.
	\end{itemize}
\end{enumerate}
\chapter{\color{blue}Words often misspelled}
\label{chap6}
	\begin{table}[h]
	\centering
	\begin{tabular}{|c|c|c|}
		\hline
		accidentally & formerly & privilege\\
		advice & humorous & pursue\\
		affect & hypocrisy & repetition\\
		beginning & immediately & rhyme\\
		believe & incidentally & rhythm\\
		benefit & latter & ridiculous\\
		challenge & led & sacrilegious\\
		criticize & lose & seize\\
		deceive & marriage & separate\\
		definite & mischief & shepherd\\
		describe & murmur & siege\\
		despise & necessary & similar\\
		develop & occurred & simile\\
		disappoint & parallel & too\\
		duel & Philip & tragedy\\
		ecstasy & playwright & tries\\
		effect & preceding & undoubtedly\\
		existence & prejudice & until\\
		fiery & principal &\\
		\hline	
	\end{tabular}
\end{table}

\chapter{\color{blue}Conclusion}
\label{chap7}
We write magazine, journals, thesis paper etc. But many of us don’t know the technique of writing. This book is very important for that writers. At the beginning of the book readers are introduced with some basic rules of using parts-of-speech. In Chapter \ref{chap2}, we can learn about the proper use of punctuation and conjunction. At the end of this chapter deals with the structure of sentence.Chapter \ref{chap3} describes us acqurate way of writing a composition, summary, paragraph. To make these compositions and summaries more attractive a writer should know correct representation of Headings, titles, quotations etc. and all of these are briefly describer in Chapter \ref{chap4}. In Chapter \ref{chap5} , we may concern about some common mistakes in English writing and how to deal with these mistakes. There are many words that are so hard in spelling. So In chapter \ref{chap4} we show a table \ref{chap4} of difficult words. After completing the book properly,a diligent reader should be able to build strong grammatical foundation and build his own style following some basic rules.
\end{document}}